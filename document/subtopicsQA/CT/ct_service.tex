 \begin{longtable}{p{.23\textwidth} | p{.55\textwidth} | p{.12\textwidth}}
\hline
  \textbf{Methode} & \textbf{Test} & \textbf{Ergebnis}\\
  \hline
  createSaltTest() & Überprüft die korrekte Erstellung des Salzes & true \\
  \hline
  changeTest() & Testet die Änderung der Accountdaten & true \\
  \hline
  registerTest() & Registriert einen User und überprüft die korrekte Erstellung & true \\
  \hline
  deleteTest() & Testet die korrekte Löschung eines registrierten Accounts & true \\
  \hline
  accountIdTest() & Vergleicht die Account-Id des Account Objektes mit dem dazugehörigem Wert in der Datenbank & true \\
  \hline
  authenticateTest() &  Authentifizierungstest eines bereits registrierten Accounts & true \\
  \hline
  verifyTest() & Durchläuft einen Verifizierungsvorgang eines Accounts & true \\
  \hline
  isVerifiedValidTest() & Überprüft ob einen bereits verifizierten Account verifiziert ist & true \\
  \hline
  isVerifiedFalseTest() & Überprüft ob ein nicht verifizierter Account verifiziert ist & false \\
  \hline
  invalidMailTest() & Testet die Überprüfung einer invaliden Mailadresse & false \\
  \hline
  blankMailTest() & Testet die Überprüfung einer leeren Mailadresse & false \\
  \hline
  \caption{AccountManagerTest}
 \end{longtable}
 
 \begin{longtable}{p{.23\textwidth} | p{.55\textwidth} | p{.12\textwidth}}
\hline
  \textbf{Methode} & \textbf{Test} & \textbf{Ergebnis}\\
  \hline
  videoInfoListTest() & Überprüft die korrekte Ausgabe der VideoInfoList & true \\
  \hline
  downloadTest() & Durchläuft einen Downloadvorgang und überprüft dabei ob ein Inputstream ankommt, mit dem dann eine Testdatei angelegt wird & true \\
  \hline
  deleteTest() & Legt eine Video und Metadata Datei an, die über den Testverlauf gelöscht werden sollen & true \\ 
  \hline
  metadataTest() & Testet die Korrektheit  der Metadateninfomationen, welche über den Klienten angefragt werden & true \\
  \hline
  \caption{VideoManagerTest}
 \end{longtable}
 
  \begin{longtable}{p{.23\textwidth} | p{.55\textwidth} | p{.12\textwidth}}
\hline
  \textbf{Methode} & \textbf{Test} & \textbf{Ergebnis}\\
  \hline
  registerTest() & registriert einen User & true \\
  \hline
  getAccountIdTest() & registriert einen User und vergleicht die id mit der id, die in der Datenbank eingetragen ist & true \\
  \hline
  isVerifiedTest() & registriert und verifiziert einen User & true \\
  \hline
  saveProcessedVideo\newline AndMetaTest() & speichert Video- und Metanamen eines Users nach dessen Registration & true \\
  \hline
  getVideoInfoTest() & testet die VideoInformationen nach Speichern der Video- und Metanamen eines Users und dessen Registration & true \\
  \hline
  getVideoIdBy\newline NameTest() & testet, ob die Methode die VideoId anhand des Namens zurückgibt & true \\
  \hline
  getVideoInfo\newline ListTest() & speichert mehrere Videos zu einem User und prüft, ob die Methode getVideoInfoList ein korrektes Array zurückgibt & true \\
  \hline
  deleteVideo\newline AndMetaTest() & erstellt ein Video zu einem User und löscht es wieder. Test prüft, ob Video in der Datenbank danach noch existiert & true \\
  \hline
  getMeta\newline NameTest() & Testet, ob die Methode getMetaName den Metanamen anhand der VideoId zurückgibt & true \\
  \hline
  setMailTest() & Testet, ob die Mail des aktuellen Accounts nach Aufruf der Methode geändert wurde & true \\
  \hline
  setPasswordTest() & Testet, ob das Passwort des aktuellen Accounts nach Aufruf der Methode geändert wurde & true \\
  \hline
  delete\newline AccountTest() & Testet, ob der aktuelle Account nach Aufruf der Methode noch in der Datenbank existiert & true \\
  \hline
  authenticateTest() & erstellt einen Account und testet, ob die Methode authenticate den aktuellen Account authentifiziert & true \\
  \hline
  isMail\newline ExistingTest() & Testet, ob die Mail schon in der Datenbank existiert & true \\
  \hline
  getSaltTest() & registriert einen Account mit einem bestimmten Salt, um zu testen, ob das Salt in der Datenbank richtig vorhanden ist & true \\
  \hline
  \caption{DatabaseManagerTest}
 \end{longtable}
 
 \begin{longtable}{p{.23\textwidth} | p{.55\textwidth} | p{.12\textwidth}}
\hline
  \textbf{Methode} & \textbf{Test} & \textbf{Ergebnis}\\
  \hline
  nullTest() & Ruft anonymize() mit null als Parameter auf & false\\
  \hline
  validTest() & Ruft anonymize() mit korrekten Parametern auf & true\\
  \hline
  noVideoTest() & Ruft anonymize() auf und versucht ein File zu anonymisieren, das kein Video ist & false\\
  \hline 
  \caption{OpenCVAnonymizeTest}
 \end{longtable}
 
 \begin{longtable}{p{.23\textwidth} | p{.55\textwidth} | p{.12\textwidth}}
\hline
  \textbf{Methode} & \textbf{Test} & \textbf{Ergebnis}\\
  \hline
  nullTest() & Ruft anonymize() mit null als Parameter auf & false\\
  \hline
  validTest() & Ruft anonymize() mit korrekten Parametern auf & true\\
  \hline
  noVideoTest() & Ruft anonymize() auf und versucht ein File zu anonymisieren, das kein Video ist & false\\
  \hline 
  \caption{OpenCVPythonAnonymizeTest}
 \end{longtable}
 
 \begin{longtable}{p{.23\textwidth} | p{.55\textwidth} | p{.12\textwidth}}
\hline
   \textbf{Methode} & \textbf{Test} & \textbf{Ergebnis}\\
  \hline
  nullTest() & Ruft decrypt() mit null als Parameter auf & false\\
  \hline
  validTest() & Ruft decrypt() mit korrekten Parametern auf & true\\
  \hline 
  \caption{AESDecryptorTest}
 \end{longtable}
 
 \begin{longtable}{p{.23\textwidth} | p{.55\textwidth} | p{.12\textwidth}}
 \hline
 \textbf{Methode} & \textbf{Test} & \textbf{Ergebnis}\\
  \hline
  nullTest() & Ruft decrypt() mit null als Parameter auf & null\\
  \hline
  validTest() & Ruft decrypt() mit korrekten Parametern auf & SecretKey mit Algorighmus 'AES'\\
  \hline 
  invalidFileTest() & Ruft decrypt() auf und versucht ein File zu entschlüsseln, das keinen SecretKey enthält & null\\
  \hline
  \caption{RSADecryptorTest}
 \end{longtable}
 
 \begin{longtable}{p{.23\textwidth} | p{.55\textwidth} | p{.12\textwidth}}
\hline
   \textbf{Methode} & \textbf{Test} & \textbf{Ergebnis}\\
  \hline
  nullTest() & Ruft decrypt() mit null als Parameter auf & false\\
  \hline
  validTest() & Ruft decrypt() mit korrekten Parametern auf & true\\
  \hline 
  \caption{DecryptTest}
 \end{longtable}
 
 \begin{longtable}{p{.23\textwidth} | p{.55\textwidth} | p{.12\textwidth}}
\hline
  \textbf{Methode} & \textbf{Test} & \textbf{Ergebnis}\\
  \hline
  emptyChainTest() & Startet eine VideoProcessingChain mit Chain.EMPTY (Testet Speichern und Löschen von tmp-Daten) & true\\
  \hline
  simpleChainTest() & Startet eine VideoProcessingChain mit Chain.SIMPLE (Testet Chain ohne Anonymisierung) & true\\
  \hline
  normalChainTest() & Startet eine VideoProcessingChain mit Chain.NORMAL (Testet vollständige Chain) & true\\
  \hline
  pythonChainTest() & Startet eine VideoProcessingChain mit Chain.PYTHON (Testet Chain vollständige Chain mit Python Anonymisierung) & true\\
  \hline
  \caption{VideoProcessingChainTest}
 \end{longtable}
 
 \begin{longtable}{p{.23\textwidth} | p{.55\textwidth} | p{.12\textwidth}}
\hline
  \textbf{Methode} & \textbf{Test} & \textbf{Ergebnis}\\
  \hline
  addNullTaskTest() & Ruft addTask() mit Null als Parameter auf & "Not all inputs were given correctly"\\
  \hline
  addValidTaskTest() & Ruft addTask mit korrekten Parametern auf & "Finished editing Video"\\
  \hline
  shutdownTest() & Ruft shutdown auf und versucht direkt danach einen Task hinzuzufügen & "Processing module is shut down"\\
  \hline
  \caption{VideoProcessingManagerTest}
 \end{longtable}