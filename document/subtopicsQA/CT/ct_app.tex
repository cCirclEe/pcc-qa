%%%%%%%%%%%%%%%%%%%%%%%%%%%%%%%%%%%%%%%%%%%%%%%%%%%%%%%%
%                    Memory Manager                    %
%%%%%%%%%%%%%%%%%%%%%%%%%%%%%%%%%%%%%%%%%%%%%%%%%%%%%%%%
\begin{longtable}{p{.23\textwidth} | p{.55\textwidth} | p{.12\textwidth}}
\hline
  \textbf{Methode} & \textbf{Test} & \textbf{Ergebnis}\\
  \hline
  settingsTest() & speichere fertiges Settings-Objekt mit saveSettings() und lese mit getSettings() aus & identisches Settings-Objekt \\
  \hline
  accountDataTest() & speichere fertiges AccountData-Objekt mit saveAccountData() und lese mit getAccountData() aus & identisches AccountData-Objekt \\
  \hline
  deleteAccount-\newline DataTest() & speichere mit saveAccountData() ein AccountData-Objekt und lösche es mit deleteAccountData() & File existiert nicht mehr \\
  \hline
  getTempVideo-\newline FileTest() & erstelle ein File im tempVideoFile-Directory und rufe getTempVideoFile() auf & File nicht null \\
  \hline
  getTempMetaData-\newline 
  FileTest() & erstelle ein File im tempMetaData-Directory und rufe getTempMetaDataFile() auf & true \\
  \hline
  deleteEncrypted-\newline SymmetricKey-\newline FileTest() & erstelle ein File im korrekten Pfad und lösche es mit deleteEncryptedSymmetricKeyFile & File existiert nicht mehr \\
  \hline
  deleteEncrypted-\newline MetadataFileTest() & erstelle ein File im korrekten Pfad und lösche es mit deleteEncryptedMetadataFile() & File existiert nicht mehr \\
  \hline
  deleteReadable-\newline MetadataTest() & erstelle ein File im korrekten Pfad und lösche es mit deleteReadableMetadata() & File existiert nicht mehr \\
  \hline
  deleteEncrypted-\newline VideoFileTest() & erstelle ein File im korrekten Pfad und lösche es mit deleteEncryptedVideoFile() & File existiert nicht mehr \\
  \hline
  createEncrypted-\newline SymmetricKey-\newline FileTest() & Erstelle eine Datei im korrekten Pfad & Dateipfad und Name sind korrekt \\
  \hline
  createEncrypted\newline VideoFileTest() & rufe createEncryptedVideoFile() auf & Dateipfad und Name sind korrekt \\
  \hline
  createEncrypted-\newline MetaFileTest() & rufe createEncryptedMetaFile() auf & Dateipfad und Name sind korrekt \\
  \hline
  createReadable\newline MetadataFileTest() & rufe createReadableMetadataFile() auf & Dateipfad und Name sind korrekt \\
  \hline
  getAll\newline VideosTest() & Erstelle Beispielvideos und rufe getAllVideos() auf & List enthält alle Videos \\
  \hline
  getEncrypted-\newline Symmetric\newline KeyTest() & erstelle ein File und hole es mit getEncryptedSymmetricKey() & File ist nicht null \\
  \hline
  getEncrypted-\newline VideoTest() & erstelle ein File und hole es mit getEncryptedVideo() & File ist nicht null \\
  \hline
  getReadable-\newline MetadataTest() & erstelle ein File und hole es mit getReadableMetadata() & File ist nicht null \\
  \hline
  \caption{MemoryManagerTest}
 \end{longtable}
 
%%%%%%%%%%%%%%%%%%%%%%%%%%%%%%%%%%%%%%%%%%%%%%%%%%%%%%%%
%                    AuthenticateTaskTest			  %
%%%%%%%%%%%%%%%%%%%%%%%%%%%%%%%%%%%%%%%%%%%%%%%%%%%%%%%% 
 
\begin{longtable}{p{.23\textwidth} | p{.55\textwidth} | p{.12\textwidth}}
\hline
  \textbf{Methode} & \textbf{Test} & \textbf{Ergebnis}\\
  \hline
  missingTest() & Überprüft Authentifizierungswert eines nicht existenten Accounts & FAILURE MISSING \\
  \hline
  successTest() & Überprüft Authentifizierungswert eines registrierten und verifizierten Accounts & SUCCESS \\
  \hline
  notVerifiedTest() & Überprüft den Authentifizierungswert eines registrierten und nicht verifizierten Accounts & NOT VERIFIED \\
  \hline
  missmatchTest() & Überprüft den Authentifizierungswert eines registrierten und verifizierten Accounts mit falscher Passwordeingabe & FAILURE MISSMATCH \\
  \hline
  failureOtherTest() & Überprüft den Authentifizierungswert beim Übergeben eines Null-Objektes & FAILURE OTHER \\
  \hline
   onPostExecute- \newline NoNetworkTest() & Überprüft die onPostExecute() Methode mit der Eingabe des States FAILURE NETWORK & "No network available" \\
  \hline
  onPostExecute- \newline SuccessTest() & Überprüft die onPostExecute() Methode mit der Eingabe des States SUCCESS & SUCCESS \\
  \hline
  \caption{AuthenticateTaskTest}
 \end{longtable}
 
%%%%%%%%%%%%%%%%%%%%%%%%%%%%%%%%%%%%%%%%%%%%%%%%%%%%%%%%
%                    VideoUploadTaskTest               %
%%%%%%%%%%%%%%%%%%%%%%%%%%%%%%%%%%%%%%%%%%%%%%%%%%%%%%%%
\begin{longtable}{p{.23\textwidth} | p{.55\textwidth} | p{.12\textwidth}}
\hline
  \textbf{Methode} & \textbf{Test} & \textbf{Ergebnis}\\
  \hline
  Account- \newline FailureTest() & Überprüft den Rückgabewert des Services bei der Übergabe eines nicht existenten Accounts & ACCOUNT FAILURE \\
  \hline
  uploadValidTest() & Überprüft den Rückgabewert des Services bei korrekter Angabe aller benötigten Parameter für einen Upload  & SUCCESS \\
  \hline
  notVerifiedTest() & Überprüft den Authentifizierungswert eines registrierten und nicht verifizierten Accounts & NOT VERIFIED \\
  \hline
  failureOtherTest() & Überprüft den Rückgabewert des Services beim Übergeben eines NullObjektes(JSON mit Accountdaten) & FAILURE OTHER \\
  \hline
   onPostExecute- \newline NoNetworkTest() & Überprüft die onPostExecute() Methode mit der Eingabe des States FAILURE NETWORK & "No network available" \\
  \hline
  onPostExecute- \newline SuccessTest() & Überprüft die onPostExecute() Methode mit der Eingabe des States SUCCESS & SUCCESS \\
  \hline
  \caption{VideoUploadTaskTest}
 \end{longtable}
 
%%%%%%%%%%%%%%%%%%%%%%%%%%%%%%%%%%%%%%%%%%%%%%%%%%%%%%%%
%                    RSAEncryptorTest               %
%%%%%%%%%%%%%%%%%%%%%%%%%%%%%%%%%%%%%%%%%%%%%%%%%%%%%%%%
\begin{longtable}{p{.23\textwidth} | p{.55\textwidth} | p{.12\textwidth}}
\hline
  \textbf{Methode} & \textbf{Test} & \textbf{Ergebnis}\\
  \hline
  nullTest() & Prüft ob bei Übergabe von Null ein Fehlschlag zurückgegeben wird & Nicht Verschlüsselt \\
  \hline
  validTest() & Überprüft ob eine Datei verschlüsselt und gespeichert wird & Verschlüsselt \\
  \hline
  \caption{RSAEncryptorTest}
 \end{longtable}
 
%%%%%%%%%%%%%%%%%%%%%%%%%%%%%%%%%%%%%%%%%%%%%%%%%%%%%%%%
%                    VideoRingBufferTest               %
%%%%%%%%%%%%%%%%%%%%%%%%%%%%%%%%%%%%%%%%%%%%%%%%%%%%%%%%
\begin{longtable}{p{.23\textwidth} | p{.55\textwidth} | p{.12\textwidth}}
\hline
  \textbf{Methode} & \textbf{Test} & \textbf{Ergebnis}\\
  \hline
  bufferValidation() & Prüft ob die Kapazität des Buffers dem übergebenen Wert entspricht & Gleiche Kapazität \\
  \hline
  putCapacity() & Prüft ob alle eingefügten Dateien ohne überlauf des Buffers gespeichert werden & Dateien vorhanden\\
  \hline
  putMoreThan- \newline Capacity() & Prüft ob alle eingefügten Dateien mit überlauf des Buffers gespeichert werden und die Ältesten entfernt werden & Dateien vorhanden\\
  \hline
  popCapactity() & Prüft ob Dateien einzeln abgefragt werden können & Dateien abfragbar \\
  \hline
  popMoreThan- \newline Capacity() & Prüft ob pop() nach dem leeren des Buffers null zurückgibt & Null zurückgegeben\\
  \hline
  destroy() & Prüft ob durch destroy() flushAll() aufgerufen wird & Alle Dateien gelöscht \\
  \hline
  flushAll() & Prüft ob durch destroy() alle Dateien gelöscht werden & Alle Dateien gelöscht \\
  \hline
  \caption{VideoRingBufferTest}
 \end{longtable}
 
%%%%%%%%%%%%%%%%%%%%%%%%%%%%%%%%%%%%%%%%%%%%%%%%%%%%%%%%
%                    AsyncPersistorTest               %
%%%%%%%%%%%%%%%%%%%%%%%%%%%%%%%%%%%%%%%%%%%%%%%%%%%%%%%%
\begin{longtable}{p{.23\textwidth} | p{.55\textwidth} | p{.12\textwidth}}
\hline
  \textbf{Methode} & \textbf{Test} & \textbf{Ergebnis}\\
   \hline
   sleepEven() & Prüft ob bei gerader Videolänge die Persistierung zum richtigen Zeitpunkt ausgelöst wird & onPersistingStarted() aufgerufen \\
   \hline
   sleepUneven() & Prüft ob bei ungerader Videolänge die Persistierung zum richtigen Zeitpunkt ausgelöst wird & onPersistingStarted() aufgerufen \\
   \hline
   waitForUi() & Prüft ob der Persistor vor dem Speichern auf die Bestätigung der UI wartet & onPersistingStopped() nicht aufgerufen \\
   \hline
   nonWritable- \newline ReadableMetadata() & Prüft ob die Persistierung abgebrochen wird wenn die ReadableMetadata-Datei nicht beschreibbar ist & Persistierung abgebrochen \\
  \hline
   nonWritable- \newline EncryptedMeta() & Prüft ob die Persistierung abgebrochen wird wenn die EncryptedMetadata-Datei nicht beschreibbar ist & Persistierung abgebrochen \\
  \hline
   nonWritable- \newline SymmetricKey() & Prüft ob die Persistierung abgebrochen wird wenn die SymmetricKey Datei nicht beschreibbar ist & Persistierung abgebrochen \\
  \hline
   noVideoSnippets() & Prüft ob die Persistierung abgebrochen wird wenn keine Video-Stücke übergeben werden & Persistierung abgebrochen \\
  \hline
   nonWritableTemp- \newline Video() & Prüft ob die Persistierung abgebrochen wird wenn die temporäre Datei nicht beschreibbar ist & Persistierung abgebrochen \\
  \hline
   nonWritable- \newline EncryptedVideo() & Prüft ob die Persistierung abgebrochen wird wenn die Zieldatei nicht beschreibbar ist & Persistierung abgebrochen \\
  \hline
   doInBackground() & Prüft ob die Persistierung bei korrekten Daten korrekt abläuft & Persistierung erfolgreich \\
  \hline
   onPostExecute() & Prüft ob Ergebnisse des Hintergrundtasks weitergeleitet werden & Persistierung abgebrochen \\
  \hline
   nonWritable- \newline EncryptedVideo() & Prüft ob die Persistierung abgebrochen wird wenn keine Metadaten übergeben werden & Persistierung abgebrochen \\
  \hline
  \caption{AsyncPersistorTest}
 \end{longtable}
 
%%%%%%%%%%%%%%%%%%%%%%%%%%%%%%%%%%%%%%%%%%%%%%%%%%%%%%%%
%                    CompatCameraHandlerLifecycleTest               %
%%%%%%%%%%%%%%%%%%%%%%%%%%%%%%%%%%%%%%%%%%%%%%%%%%%%%%%%
\begin{longtable}{p{.23\textwidth} | p{.55\textwidth} | p{.12\textwidth}}
\hline
  \textbf{Methode} & \textbf{Test} & \textbf{Ergebnis}\\
   \hline
   inOrder() & Prüft ob bei richtiger Aufrufreihenfolge keine Exception aufgerufen wird & Keine Fehler \\
   \hline
   noLifecycle() & Prüft ob eine IllegalStateException geworfen wird wenn der Lifecycle nicht beachtet wird & Exception geworfen \\
   \hline
   resumeBefore- \newline Create() & Prüft ob eine IllegalStateException geworfen wird wenn der Lifecycle nicht beachtet wird & Exception geworfen \\
   \hline
   pauseBefore- \newline Create() & Prüft ob eine IllegalStateException geworfen wird wenn der Lifecycle nicht beachtet wird & Exception geworfen \\
   \hline
   pauseBefore- \newline Resume() & Prüft ob eine IllegalStateException geworfen wird wenn der Lifecycle nicht beachtet wird & Exception geworfen \\
   \hline
   destroyBefore- \newline Create() & Prüft ob eine IllegalStateException geworfen wird wenn der Lifecycle nicht beachtet wird & Exception geworfen \\
   \hline
   destroyBefore- \newline Resume() & Prüft ob eine IllegalStateException geworfen wird wenn der Lifecycle nicht beachtet wird & Exception geworfen \\
   \hline
  \caption{CompatCameraHandlerLifecycleTest}
 \end{longtable}
 
\newpage
%%%%%%%%%%%%%%%%%%%%%%%%%%%%%%%%%%%%%%%%%%%%%%%%%%%%%%%%
%                    CompatCameraHandlerTest               %
%%%%%%%%%%%%%%%%%%%%%%%%%%%%%%%%%%%%%%%%%%%%%%%%%%%%%%%%
\begin{longtable}{p{.23\textwidth} | p{.55\textwidth} | p{.12\textwidth}}
\hline
  \textbf{Methode} & \textbf{Test} & \textbf{Ergebnis}\\
   \hline
   schedulePersisting() & Prüft ob der Recordcallback über Start und Ende der Aufnahme benachrichtigt wird & Benachrichtigt \\
   \hline
   schedulePersisting- \newline Twice() & Prüft ob mehrfache Aufrufe von schedulePersisting() ignoriert werden & Benachrichtigt \\
   \hline
   updateMetadata() & Prüft ob der Recordcallback über Start, Ende und Fehler benachrichtigt wird wenn die Metadaten null sind & Benachrichtigt \\
   \hline
  \caption{CompatCameraHandlerTest}
\end{longtable}
 
%%%%%%%%%%%%%%%%%%%%%%%%%%%%%%%%%%%%%%%%%%%%%%%%%%%%%%%%
%                    CompatCameraHandlerTest               %
%%%%%%%%%%%%%%%%%%%%%%%%%%%%%%%%%%%%%%%%%%%%%%%%%%%%%%%%
\begin{longtable}{p{.23\textwidth} | p{.55\textwidth} | p{.12\textwidth}}
\hline
  \textbf{Methode} & \textbf{Test} & \textbf{Ergebnis}\\
   \hline
   onClickSingle() & Prüft ob die Aufnahme nicht bei einzelnen Touch-Eingaben ausgelöst wird & Nicht ausgelöst \\
   \hline
   onClickDelayed() & Prüft ob die Aufnahme nicht bei einzelnen Touch-Eingaben lange nacheinander ausgelöst wird & Nicht ausgelöst \\
   \hline
   onClickDouble() & Prüft ob die Aufnahme nicht bei einem Doppelklick ausgelöst wird & Ausgelöst \\
   \hline
  \caption{CompatCameraHandlerTest}
\end{longtable}
                                                      
%%%%%%%%%%%%%%%%%%%%%%%%%%%%%%%%%%%%%%%%%%%%%%%%%%%%%%%%
%                    Your Own                          %
%%%%%%%%%%%%%%%%%%%%%%%%%%%%%%%%%%%%%%%%%%%%%%%%%%%%%%%%
