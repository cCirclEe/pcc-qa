\section{System-Tests}

\subsection{Einleitung}
Der Systemtest ist ein Teil der Testphase unseres Produktes, bei der das gesamte System gegen die gesamten Anforderungen (funktionale und nicht-funktionale Anforderungen) getestet wird. Fur gewöhnlich findet der Test auf einer Testumgebung statt und wird mit Testdaten durchgeführt. Die Testumgebung soll reale Bedingungen simulieren, bzw nahe an diese herankommen.
\subsection{Funktionale Systemtests}
Wir kümmern uns in diesem Abschnitt nur um die Funktionalen Systemtests.
Wir haben die jar- Datei des Webservice auf einem unserer privaten Server auf dem Port 2222 gestartet. Zudem haben wir auf dem selben Server das Webinterface auf dem Port 9999 gestartet. Somit haben wir eine reale Umgebung geschaffen, welche über das Internet erreichbar ist. Damit kann auch die App über das Internet mit dem Service kommunizieren. Wir haben alle Komponenten auf die Umgebung angepasst und angefangen, die funktionalen Anforderungen des Pflichtenhefts zu testen. \\

Die Analyse der eingebauten funktionalen Anforderungen sind an das Kapitel \textbf{\eqref{funcspec}} angehängt.
Darüber hinaus haben wir das Zusammenspiel der Komponenten getestet. \newline 

Hier ein Beispielablauf:  \\

Zuerst haben wir nach dem ``Einloggen'' auf der App einen Unfall auf der App aufgenommen, welcher auf in der App unter ``Videos'' angezeigt wurde. Nachdem die Funktionen ``Löschen'' und ``Metadaten anzeigen'' erfolgreich getestet wurden, haben wir den ``Upload''-Button betätigt. Das Video wurde an den Webserver geschickt, was wir in unseren Logs mitverfolgen konnten. Das Video wurde auf dem Webserver erfolgreich entschlüsselt, anonymisiert und danach in der Datenbank zum User hinzugefügt. Nun haben wir das Webinterface im Browser geöffnet und uns eingeloggt. Nun haben wir diekt beim Starten die Ansicht unserer hochgeladenen Videos. Unser Video wird auch hier gelistet. Die Metadaten stimmen und auch das Löschen ist erfolgreich. Mit dem ``Download''-Button wird das Video heruntergeladen. Das Video konnte in einem entsprechenden Video-Player angesehen werden.


