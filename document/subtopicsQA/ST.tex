\section{System-Tests}

\subsection{Einleitung}
Der Systemtest ist ein Teil der Testphase unseres Produktes, bei der das gesamte System gegen die gesamten Anforderungen (funktionale und nicht-funktionale Anforderungen) getestet wird. Fur gewöhnlich findet der Test auf einer Testumgebung statt und wird mit Testdaten durchgeführt. Die Testumgebung soll reale Bedingungen simulieren, bzw nahe an diese herankommen.
\subsection{Funktionale Systemtests}
Wir kümmern uns in diesem Abschnitt nur um die Funktionalen Systemtests.
Wir haben die jar- Datei des Webservice auf einem unserer privaten Server auf dem Port 2222 gestartet. Zudem haben wir auf dem selben Server das Webinterface auf dem Port 9999 gestartet. Somit haben wir eine reale Umgebung geschaffen, welche über das Internet erreichbar ist. Damit kann auch die App über das Internet mit dem Service kommunizieren. Wir haben alle Komponenten auf die Umgebung angepasst und angefangen, die funktionalen Anforderungen des Pflichtenhefts zu testen. \\

\begin{longtable}{p{.13\textwidth} | p{.4\textwidth} | p{.2\textwidth} | p{.12\textwidth}}
\hline
  \textbf{Test} & \textbf{Inhalt} & \textbf{beteiligte\newline Komponente} & \textbf{Ergebnis}\\
  \hline
  PK1000 & Nutzer müssen sich anmelden, um die App zu verwenden & App, Service & JA \\
  \hline
  PK1010 & Nur registrierte Nutzer können die App verwenden. & App, Service & JA \\
  \hline 
  PK1020 & Der Straßenverkehr wird durch die Smartphonekamera beobachtet. & App & JA
  \\
  \hline 
  PK1030 & Relevante Videodaten werden verschlüsselt abgespeichert. & App & JA
  \\
  \hline 
  PK1040 & Relevante Daten werden durch Auswertung der Sensordaten des Smartphones erkannt. Hierbei werden die Werte des G-Sensors ausgewertet. & App & JA
  \\
  \hline 
  PK1050 & Die Aufnahme kann manuell gestartet werden, auch wenn der G-Sensor des Smartphones keinen Anlass dazu gibt. & App & JA
  \\
  \hline 
  PK1060 & Während der Aufnahme werden sämtliche Nutzereingaben und G-Sensordaten ignoriert. & App & JA
  \\
  \hline 
  PK1070 & Es werden relevante Metadaten mit den Videodaten abgespeichert. & App & JA
  \\
  \hline 
  PK1080 & Es wird ab dem Appstart mit dem Beobachten des Straßenverkehrs begonnen. & App & JA (Anmeldung)
  \\
  \hline 
  PK1090 & Es wird nur das Hochformat unterstützt. & App & JA
   \\
  \hline 
  PK1100 & Die Beobachtung läuft nur während sich die App im Vordergrund befindet. & App & JA
   \\
  \hline 
  PK1110 & Videodaten werden verschlüsselt, sobald sie persistiert werden. & App & JA
   \\
  \hline 
  PK1120 & Verschlüsselte Videodaten werden aufgelistet. & App & JA
   \\
  \hline 
  PK1130 & Verschlüsselte Videodaten können gelöscht werden. & App & JA
   \\
  \hline 
  PK1140 & Vom Nutzer ausgewählte verschlüsselte Videodaten werden an einen Web-Dienst gesendet, der diese anonymisiert. & App, Service & JA
   \\
  \hline 
  PK1150 & Geräte, auf denen Android API Level 19 (Android 4.4) und höher läuft werden unterstützt. & App & JA
   \\
  \hline 
  PK1160 & Die Benutzeroberfläche wird für Geräte ab einer Displaydiagonale von 4 Zoll optimiert. & App & JA
   \\
  \hline 
  PK1170 & Wenn verschlüsselte Videodaten lange Zeit nicht zum Anonymisieren ausgewählt wurden, wird der Nutzer benachrichtigt, dass er diese löschen kann. & App & NEIN
   \\
  \hline 
  PK1180 & Die aufnahmespezifischen Einstellungen werden angezeigt. & App & JA
   \\
  \hline 
  PK1190 & Die Standardsprache ist Deutsch. & App & JA
   \\
  \hline 
  PK2000 & Es existiert eine Schnittstelle, um Videodaten hochzuladen. & App & JA
   \\
  \hline 
  PK2010 & Von der App gesendete Videodaten werden anonymisiert. & App & JA
   \\
  \hline 
  PK2020 & Nach Abschluss der Anonymisierung wird der Nutzer per E-Mail benachrichtigt. & App, Service & NEIN (Rückmeldung in App)
   \\
  \hline 
  PK2030 & Es existiert eine Schnittstelle, um Nutzeraccounts anzulegen. & Webinterface & JA
   \\
  \hline 
  PK2040 & Es existiert eine Schnittstelle, um Nutzeraccounts zu verwalten. & Webinterface & JA
   \\
  \hline 
  PK2050 & Es existiert eine Schnittstelle, um die Videodaten eines Nutzers verwalten
zu können. & App, Webinterface & JA
   \\
  \hline 
  PK2060 & Nutzer müssen ihre E-Mail-Adresse verifizieren, um sich anmelden zu können. & Webinterface & JA
   \\
  \hline 
  PK2070 & Die Kommunikation zwischen App und Web-Dienst wird durch eine REST-API realisiert. & App, Service & JA
   \\
  \hline 
  PK2080 & Die Kommunikation zwischen Web-Interface und Web-Dienst wird durch eine REST-API realisiert. & Webinterface, Service & JA
   \\
  \hline 
  PK2090 & Es existiert eine obere Schranke für die Anzahl der Videodaten, die ein Nutzer zur gleichen Zeit auf seinem Nutzeraccount online speichern kann. & Service & NEIN
   \\
  \hline 
  PK2100 & Passwörter werden nur als Hash-Code abgespeichert. & Service & JA
   \\
  \hline 
  PK2110 & Es wird Jetty verwendet. & Service & JA
   \\
  \hline 
  PK3000 & Es können Nutzeraccounts angelegt werden. & Webinterface, Service & JA
   \\
  \hline 
  PK3010 & Es können Nutzeraccounts verwaltet werden. & Webinterface, Service & JA
   \\
  \hline 
  PK3020 & Es können Videodaten verwaltet werden. & Webinterface, Service & JA
   \\
  \hline 
  PK3030 & Es können Videodaten heruntergeladen werden. & Webinterface, Service & JA
   \\
  \hline 
  PK3040 & Nur eingeloggte Nutzer haben Zugriff auf ihre Nutzerdaten. & App, Webinterface, Service & JA
   \\
  \hline 
  PK3050 & Es können Passwort und E-Mail-Adresse geändert werden. & Webinterface, Service & JA
   \\
  \hline 
  PK3060 & Die Standardsprache ist Deutsch. & App, Webinterface & JA
  \\
  \hline
  \caption{Funktionale Anforderungen des Pflichtenhefts}
 \end{longtable}
