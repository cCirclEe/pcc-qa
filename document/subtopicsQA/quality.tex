\chapter{Qualität} \label{Qualitaet}
In diesem Kapitel wollen wir auf die Frage der Qualität eingehen. Nun, was ist Qualität? Qualität hat viele Gesichter, insbesondere wird jedes Produkt anhand
anderer Merkmale bewertet. Diese Merkmale ergeben sich im Groben aus dem für das Produkt vorgesehenen Verwendungszweck. Um also die Qualität der Privacy Crash Cam sinnvoll beurteilen zu können, müssen wir vorerst einen Qualitätsbegriff diesbezüglich definieren. Darum wollen wir im Folgenden näher darauf eingehen, welche Ziele wir für unser Projekt gesetzt haben und dabei auf die Merkmale eingehen, durch die sich die Qualität unseres
Softwareprodukts manifestiert. \\
\par
Nach ISO 9126 hängt die Qualität eines Softwareprojekts von folgenden sechs Kriterien ab:

\begin{itemize}
	\item[\textbf{Funktionalität}] Bei diesem Kriterum wird betrachtet, ob der Funktionsumfang des Programms angemessen ist. Das heißt: Reichen die vom Produkt angebotenen Funktionen aus, um alle, oder zumindest die meisten, Variationen des Problems zu lösen, oder sind
es gar zu viele?
	\item[\textbf{Zuverlässigkeit}] Dieses Kriterium beantwortet die Fragen: Wie fehlerfrei arbeitet das Programm? Werden Fehler erkannt und behandelt? Werden fehlerhafte Nutzereingaben behandlet?
	\item[\textbf{Benutzbarkeit}] Benutzbarkeit bezeichnet den Bereich der Softwarequalität, in dem es darum geht, wie einfach ein Benutzer das Produkt bedienen kann. Hierbei ist vor allem die Einstiegsschwelle sowie die Erfahrungskurve zu beachten. Haben Benutzer zu Anfang Probleme mit der Software, oder brauchen sie sehr lange, um gewünschte Ergebnisse zu erzielen, zeugt
dies von schlechter Benutzbarkeit.
	\item[\textbf{Effizienz}] Bei der Effizienz geht es um den Ressourcenverbrauch der Anwendung
selbst. Messbare Kriterien sind hier z.B. Speicherverbrauch, Prozessorauslastung,
wie auch die Zeit, die das Programm zum Ermitteln einer
Lösung benötigt.
	\item[\textbf{Änderbarkeit}] Hier kommt es auf gute Wartbarkeit des Quellcodes an, d.h. wie einfach es ist, später Änderungen am Programm durchzuführen und deren Funktionen zu verifizieren.
	\item[\textbf{Übertragbarkeit}] Mit diesem Begriff ist der Aufwand beim Wechsel auf
das Produkt bezeichnet. Es sollte überprüft werden, ob sich die Software an nutzerspezifische Gegebenheiten anpassen lässt und wie leicht bzw. schnell eine Installation möglich ist.
\end{itemize}

Einige dieser Kriterien wurden bereits in vorhergehenden Dokumenten behandelt (Funktionalität + Benutzbarkeit $\rightarrow$ Pflichtenheft, Änderbarkeit + Übertragbarkeit $\rightarrow$ Entwurf). Mit den restlichen, sprich Zuverlässigkeit und Effizienz beschäftigt sich dieses Dokument in den folgenden Kapiteln.