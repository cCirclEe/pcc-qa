\section{Integration-Tests}

\subsection{Einleitung}

Als Integrations-Tests werden die Tests bezeichnet, die die korrekte Kommunikation zwischen einzelnen Komponenten überprüfen. In unserem Falle müssen also die Kommunikation zwischen App und Service, sowie zwischen Interface uns Service getestet werden.\par
Da wir auf dem Web-Service eine einheitliche REST-Schnittstelle definiert haben, können wir die Kommunikation über einen  Testclienten überprüfen, der die Aufrufe von App bzw. Interface emuliert. Die Integration-Tests sind daher alle über den ServerProxyTest im Web-Service abgedeckt. Dabei überprüft jede Methode eine REST-Schnittstellenmethode.\par 
Hierfür startet bei jedem Test ein Thread zunächst den Web-Service. Daraufhin stellt der Client-Thread seine Anfragen an den Service, der daraufhin die Anfrage bearbeitet und dem Clienten Rückmeldung gibt.

\subsection{Test-Client}

\begin{longtable}{p{.23\textwidth} | p{.55\textwidth} | p{.12\textwidth}}
\hline
  \textbf{Methode} & \textbf{Test} & \textbf{Ergebnis}\\
  \hline
   authenticate- \newline ValidTest() & Stellt eine Authentifizierungsanfrage mit einem registrierten und verifizierten Account & SUCCESS\\
  \hline
  authenticate- \newline FailTest() & Stellt eine Authentifizierungsanfrage ohne die benötigten Parameter & FAILURE \\
  \hline
  verifyTest() & Stellt eine Verifizierungsanfrage mit einem bereits registrierten Account & SUCCESS \\
  \hline
   downloadTest() & Stellt eine Download-Anfrage mit einem registrierten und verifizierten Account auf eine manuell eingesetzte Videodatei & Inputstream der Datei \\
  \hline
  videosTest() & Stellt eine Videolisten Anfrage mit einem registrierten und verifizierten Account & JSON-Objekt mit passender Videoinfo \\
  \hline
  create- \newline AccountTest() & Stellt eine Accounterstellungsanfrage eines noch nicht registrierten Accounts & SUCCESS \\
  \hline
  delete- \newline AccountTest() & Stellt eine Accountlöschungsanfrage mit einem registrierten und verifizierten Account & SUCCESS \\
  \hline
  videoDeleteTest() & Stellt eine Videolöschungsanfrage mit einem registrierten und verifizierten Account. Dabei haben wir für diesen Test eine Video und eine Metadata Datei angelegt, welche durch die Anfrage gelöscht werden sollen & SUCCESS \\
  \hline
  videoInfoTest() & Stellt eine Metadatenanfrage zu einem bereits erstellten Video mit dazugehörigen Metadaten. Der abfragende Account ist registriert und verifiziert & JSON-Objekt mit korrekten Metadaten \\
  \hline
  uploadValidTest() & Stellt eine Upload-Anfrage mit einem bereits registrierten und verifizierten Account. Die benutzen Daten sind den Anforderungen angepasst(Video und Metadatei verschlüsselt) und vollständig vorhanden (Videodatei, Metadatei und Schlüsseldatei). & "Finished editing video." \\
  \hline
  uploadFailTest() & Stellt eine Upload-Anfrage mit einem bereits registrierten und verifizierten Account. In diesem Testfall schicken wir die benötigten Parameter nicht mit & FAILURE \\
  \hline
  \caption{VideoUploadTaskTest}
 \end{longtable}