\chapter{Einleitung} \label{Einleitung}

Der Feinschliff, der in der Automobil- und Pharmaindustrie seit Jahrzehnten unter dem Begriff „Qualitätssicherung“ betrieben wird, hat seit geraumer Zeit auch einen Platz in der Entwicklung von Softwareprojekten gefunden.
Die Erkenntnis, dass ein Softwareprodukt nach Design und Implementierung nicht fertig gestellt ist, zeigt sich Jahr für Jahr durch mehr oder weniger spektakuläre Pannen, wie z.B. den Millennium-Bug, der Firmen schätzungsweise 600 Mio. US-Dollar gekostet hat. Auch wenn Fehler in unserem Projekt nicht so schwerwiegend sind, wie der Millennium-Bug, wollen wir unserem Nutzer ein qualitativ hochwertiges Produkt präsentieren, das seine Erwartungen vollständig erfüllt. \\ \par
Um einheitliche Maßstäbe zu setzen wird sich dieses Dokument zunächst mit dem Begriff der Qualität beschäftigen und daraus eine Reihe von Kriterien entwickeln, anhand denen wir unser Projekt bemessen wollen \eqref{Qualitaet}. \\
\par
Dieser Begriff wird danach noch erweitert und auf die von uns im Pflichtenheft schon genannten Anforderungen konkretisiert. Diese wurden am Anfang des Projekts festgelegt und sind somit nicht auf eventuelle Probleme in der Implementierung angepasst. Daher werden wir auch Ausführen inwiefern wir diese erfüllen konnten \eqref{funcspec}. \\
\par
Mithilfe der über unsere Qualitätskriterien definieren Metrik haben wir eine Reihe von und manuellen und automatisierten Tests erstellt. Über deren Ausführung und Ergebnisse berichtet das nächste Kapitel \eqref{Test}. \\
\par
Zuletzt wird noch auf Änderungen eingegangen, die durch diesen Prozess angestoßen wurden \eqref{Changelog}.