\chapter{Änderungen} \label{Changelog}
Sein Softwareprodukt umfassend zu testen und anhand genau festgelegter Kriterien zu bewerten ist ein guter Vorsatz. Wenn man die gewonnen Informationen nicht zum Anlass nimmt, sein Programm zu ändern und zu verbessern, ist die Arbeit jedoch wertlos. Deshalb hier eine Liste an Änderungen die durch den Testprozess angestoßen wurden.
\section{Web-Dienst}
\begin{itemize}
\item[\textbf{Laden von Ressourcen im Web-Dienst}] Damit auch aus dem gebauten Artefakt auf Ressourcen zugegriffen werden kann, wird nun über den ResourceLoader zugegriffen. Da aus einer Jar jedoch nicht auf Files zugegriffen werden kann, sondern nur auf Streams, der CascadeClassifier jedoch ein File benötigt, wird die Haarcascade vor dem anonymisieren aus der Jar kopiert.
\item[\textbf{Installation von OpenCV}] Da wir OpenCV nun über eine Bibliothek verwalten, muss der Benutzer OpenCV nicht mehr selbst installieren.
\item[\textbf{Konstruktor Metadata}] Das Laden der Informationen aus dem JSON-Objekt wurde in den try-catch-Block verschoben, um Ausnahmen aufzufangen.
\item[\textbf{Streams}] Im Web-Service werden an vielen Stellen Streams gebraucht um Resourcen zu laden. Manchmal wurde jedoch vergessen diese nach Verwendung zu schließen, wodurch Resource-Leaks entstehen konnten.
\item[\textbf{Pyhton-Skript}] Als wir das vom Frauenhofer-Institut bereitgestellte Python-Skript über einen Java-System-Call ausführen wollten, hat das Skript nach einiger Zeit blockiert. Zunächst waren wir verwundert, da das Skript, wenn man es per Konsole startet, einwandfrei funktioniert. Wie sich später herausstellte, war das Problem, dass das Skript Konsolenausgaben verwendete, die beim Aufruf über Java jedoch nie ausgegeben wurden und nach einiger Zeit den Systempuffer blockierten. Das Problem konnten wir lösen, indem wir den Puffer periodisch leeren und den Inhalt verwerfen
\item[\textbf{Inputstream ausgelagert}] Während dem Testen der verschiedenen Komponenten und insbesondere der download-Methode im ServerProxy ist uns aufgefallen, das wir die Erstellung des Inputstream des angefragten Videos in den VideoManager auslagern sollten und diesen dann direkt an die download-Methode übergibt. Dies führt zu der besseren Aufgabenteilung zwischen Proxy und VideoManager.
\end{itemize}
\section{Web-Interface}
\begin{itemize}
\item [\textbf{Überrschung}] Keine Überraschungen mehr und deshalb auch keine Änderungen :)
\end{itemize}
\section{App}
\begin{itemize}
\item [\textbf{Rückgabewerte}] Während der Implementierungsphase wurden die Kommunikationsnachrichten zwischen Web-Service und App verändert. Besondere Randfälle wurden hinzugefügt und Inhalte verändert. Diese Veränderung ist uns während der Testphase aufgefallen. Daher haben wir die Angaben korrigiert.
\item [\textbf{Wartezeit nach dem Auslösen der Persistierung}] Der Nutzer kann unter den Einstellungen eine bestimmte Videolänge festlegen. Der AsyncPersistor sollte nach dem Auslösen der Aufnahme exakt halb so lange wie die vom Nutzer festgelegte Videolänge warten. In dem AsyncPersistor ist dies nicht erfüllt, sobald der Nutzer eine ungerade Videolänge festlegt, da die Berechnung der Wartezeit die Videolänge in Sekunden durch zwei teilt und es somit zu Rundungsfehlern kommt. Das Problem wurde behoben indem erst in Millisekunden umgerechnet wird und anschließend geteilt wird.
\item [\textbf{Ringebuffergöße}] Der Ringbuffer funktioniert nur zuverlässig, wenn seine Kapazität größer als eins ist. Diese Zusicherung fehlte noch im Konstruktor und wurde eingefügt.
\item [\textbf{CameraHandler Lifecycle}] Der CameraHandler gibt einen Lifecycle vor, jedoch wird dieser durch die Implementierung des Interfaces nicht zwangsläufig umgesetzt. Daher wurde der CameraHandler in eine abstrakte Klasse umgewandelt, deren Methoden überschrieben werden können. Sie erlaubt nur das Aufrufen von Lifecycle-Methoden in der richtigen, von ihr festgelegten Reihenfolge.
\item [\textbf{Video Löschen}] Da während des Uploads ein Video nicht gelöscht werden sollte, wurde eine entsprechende Überprüfung eingefügt, die genau dies verhindert. Momentan kann immer nur entweder ein Video hochgeladen, oder ein Video gelöscht werden, aber Anpassungen dies Videospezifisch zu handhaben sind einfach einzubauen.
\item[\textbf{Fps Cap}] Überschreitet die in den Settings festgelegten Bildwiederholrate die maximal mögliche des Geräts setzen wir diese auf das Gerätemaximum. Die Angabe des Android Geräts ist jedoch in Bilder/1000 Sekunden, sodass der Wert auf 30'000 Fps statt 30 Fps gesetzt wurde. Dies haben wir korrigiert.
\end{itemize}