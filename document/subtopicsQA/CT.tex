\section{Komponenten-Tests}

\subsection{Einleitung}
Um die Funktionalität der einzelnen Komponenten sicherzustellen, haben wir eine Reihe automatisierter Tests erstellt. Zunächst wurden die alleinstehenden Module einzelnen getestet, um sicher zu stellen, ob jedes einzelne Modul in sich geschlossen funktioniert. Nachdem dies sicher gestellt werden konnte, wurden die Pakete getestet, die das Zusammenspiel der zugrunde liegenden Module bestimmen.\par
Das automatisierte Testen grafischer Oberflächen hat nur begrenzte Aussagekraft über die Funktionalität des Produktes. So kann man beispielsweise nicht herausfinden, wie das Produkt auf unterschiedlichen Displaygrößen und Geräten aussieht. Allerdings lassen sich UI-Tests anhand von Testprotokollen praktikabel durchführen. Hier bekommt der Tester auch direktes Feedback über Lesbarkeit, Design und Anordnung von Komponenten. Wegen des hohen Zeitaufwandes und des geringen Mehrwertes haben wir daher ein Testprotokoll erstellt, anhand dessen wir die UI manuell getestet haben.\par
Für den Erfolg eines Programm ist es essentiell, dass es bei korrekten Eingaben korrekte Ergebnisse liefert. Man kann jedoch mit an Sicherheit grenzender Wahrscheinlichkeit davon ausgehen, dass Nutzer nicht genau das macht, was man möchte. Daher haben wir auch eine Reihe von Tests eingefügt, die Ausnahmen und Fehleingaben behandeln.\par
Das folgende Kapitel wird beschreiben was in den jeweiligen Tests gemacht wurde und welches Ergebnis erwartet wird.

\newpage
\subsection{App}
%%%%%%%%%%%%%%%%%%%%%%%%%%%%%%%%%%%%%%%%%%%%%%%%%%%%%%%%
%                    Memory Manager                    %
%%%%%%%%%%%%%%%%%%%%%%%%%%%%%%%%%%%%%%%%%%%%%%%%%%%%%%%%
\begin{longtable}{p{.23\textwidth} | p{.55\textwidth} | p{.12\textwidth}}
\hline
  \textbf{Methode} & \textbf{Test} & \textbf{Ergebnis}\\
  \hline
  settingsTest() & speichere fertiges Settings-Objekt mit saveSettings() und lese mit getSettings() aus & identisches Settings-Objekt \\
  \hline
  accountDataTest() & speichere fertiges AccountData-Objekt mit saveAccountData() und lese mit getAccountData() aus & identisches AccountData-Objekt \\
  \hline
  deleteAccount\newline DataTest() & speichere mit saveAccountData() ein AccountData-Objekt und lösche es mit deleteAccountData() & File existiert nicht mehr \\
  \hline
  getTempVideo\newline FileTest() & erstelle ein File im tempVideoFile-Directory und rufe getTempVideoFile() auf & File nicht null \\
  \hline
  getTempMetaData\newline FileTest() & erstelle ein File im tempMetaData-Directory und rufe getTempMetaDataFile() auf & true \\
  \hline
  deleteEncrypted\newline SymmetricKey\newline FileTest() & erstelle ein File im korrekten Pfad und lösche es mit deleteEncryptedSymmetricKeyFile & File existiert nicht mehr \\
  \hline
  deleteEncrypted\newline MetadataFileTest() & erstelle ein File im korrekten Pfad und lösche es mit deleteEncryptedMetadataFile() & File existiert nicht mehr \\
  \hline
  deleteReadable\newline MetadataTest() & erstelle ein File im korrekten Pfad und lösche es mit deleteReadableMetadata() & File existiert nicht mehr \\
  \hline
  deleteEncrypted\newline VideoFileTest() & erstelle ein File im korrekten Pfad und lösche es mit deleteEncryptedVideoFile() & File existiert nicht mehr \\
  \hline
  createEncrypted\newline SymmetricKey\newline FileTest() & Erstelle eine Datei im korrekten Pfad & Dateipfad und Name sind korrekt \\
  \hline
  createEncrypted\newline VideoFileTest() & Erstelle eine Datei im korrekten Pfad & Dateipfad und Name sind korrekt \\
  \hline
  createEncrypted\newline MetaFileTest() & Erstelle eine Datei im korrekten Pfad & Dateipfad und Name sind korrekt \\
  \hline
  createReadable\newline MetadataFileTest() & Erstelle eine Datei im korrekten Pfad & Dateipfad und Name sind korrekt \\
  \hline
  getAll\newline VideosTest() & Erstelle Beispielvideos und rufe getAllVideos() auf & List enthält alle Videos \\
  \hline
  getEncrypted\newline Symmetric\newline KeyTest() & erstelle ein File und hole es mit getEncryptedSymmetricKey() & File ist nicht null \\
  \hline
  getEncrypted\newline VideoTest() & erstelle ein File und hole es mit getEncryptedVideo() & File ist nicht null \\
  \hline
  getReadable\newline MetadataTest() & erstelle ein File und hole es mit getReadableMetadata() & File ist nicht null \\
  \hline
  \caption{MemoryManagerTest}
 \end{longtable}
 
%%%%%%%%%%%%%%%%%%%%%%%%%%%%%%%%%%%%%%%%%%%%%%%%%%%%%%%%
%                    AuthenticateTaskTest			  %
%%%%%%%%%%%%%%%%%%%%%%%%%%%%%%%%%%%%%%%%%%%%%%%%%%%%%%%% 
 
\begin{longtable}{p{.23\textwidth} | p{.55\textwidth} | p{.12\textwidth}}
\hline
  \textbf{Methode} & \textbf{Test} & \textbf{Ergebnis}\\
  \hline
  MissingTest() & Überprüft den Authentifizierungswert eines nicht existenten Accounts & FAILURE MISSING \\
  \hline
  SuccessTest() & Überprüft den Authentifizierungswert eines registrierten und verifizierten Accounts & SUCCESS \\
  \hline
  NotVerifiedTest() & Überprüft den Authentifizierungswert eines registrierten und nicht verifizierten Accounts & NOT VERIFIED \\
  \hline
  MissmatchTest() & Überprüft den Authentifizierungswert eines registrierten und verifizierten Accounts mit falscher Passwordeingabe & FAILURE MISSMATCH \\
  \hline
  FailureOtherTest() & Überprüft den Authentifizierungswert beim Übergeben eines NullObjektes(JSON mit Accountdaten) & FAILURE OTHER \\
  \hline
  NoNetworkTest() & Überprüft die onPostExecute() Methode mit der Eingabe des States FAILURE NETWORK & "No network available" \\
  \hline
  RequestTest() & Überprüft die onPostExecute() Methode mit der Eingabe des States SUCCESS & SUCCESS \\
  \caption{AuthenticateTaskTest}
 \end{longtable}
 
%%%%%%%%%%%%%%%%%%%%%%%%%%%%%%%%%%%%%%%%%%%%%%%%%%%%%%%%
%                    VideoUploadTaskTest               %
%%%%%%%%%%%%%%%%%%%%%%%%%%%%%%%%%%%%%%%%%%%%%%%%%%%%%%%%
\begin{longtable}{p{.23\textwidth} | p{.55\textwidth} | p{.12\textwidth}}
\hline
  \textbf{Methode} & \textbf{Test} & \textbf{Ergebnis}\\
  \hline
  Account- \newline FailureTest() & Überprüft den Rückgabewert des Services bei der Übergabe eines nicht existenten Accounts & ACCOUNT FAILURE \\
  \hline
  UploadValidTest() & Überprüft den Rückgabewert des Services bei korrekter Angabe aller benötigten Parameter für einen Upload  & SUCCESS \\
  \hline
  NotVerifiedTest() & Überprüft den Authentifizierungswert eines registrierten und nicht verifizierten Accounts & NOT VERIFIED \\
  \hline
  FailureOtherTest() & Überprüft den Rückgabewert des Services beim Übergeben eines NullObjektes(JSON mit Accountdaten) & FAILURE OTHER \\
  \hline
  NoNetworkTest() & Überprüft die onPostExecute() Methode mit der Eingabe des States FAILURE NETWORK & "No network available" \\
  \hline
  onPostExecute- \newline SuccessTest() & Überprüft die onPostExecute() Methode mit der Eingabe des States SUCCESS & SUCCESS \\
  \caption{VideoUploadTaskTest}
 \end{longtable}
 
 
                                                      
%%%%%%%%%%%%%%%%%%%%%%%%%%%%%%%%%%%%%%%%%%%%%%%%%%%%%%%%
%                    Your Own                          %
%%%%%%%%%%%%%%%%%%%%%%%%%%%%%%%%%%%%%%%%%%%%%%%%%%%%%%%%

\newpage
\subsection{Webservice}
 \begin{longtable}{p{.23\textwidth} | p{.55\textwidth} | p{.12\textwidth}}
\hline
  \textbf{Methode} & \textbf{Test} & \textbf{Ergebnis}\\
  \hline
  createSaltTest() & Überprüft die korrekte Erstellung des Salzes & true \\
  \hline
  changeTest() & Testet die Änderung der Accountdaten & true \\
  \hline
  registerTest() & Registriert einen User und überprüft die korrekte Erstellung & true \\
  \hline
  deleteTest() & Testet die korrekte Löschung eines registrierten Accounts & true \\
  \hline
  accountIdTest() & Vergleicht die Account-Id des Account Objektes mit dem dazugehörigem Wert in der Datenbank & true \\
  \hline
  authenticateTest() &  Authentifizierungstest eines bereits registrierten Accounts & true \\
  \hline
  verifyTest() & Durchläuft einen Verifizierungsvorgang eines Accounts & true \\
  \hline
  isVerifiedValidTest() & Überprüft ob einen bereits verifizierten Account verifiziert ist & true \\
  \hline
  isVerifiedFalseTest() & Überprüft ob ein nicht verifizierter Account verifiziert ist & false \\
  \hline
  invalidMailTest() & Testet die Überprüfung einer invaliden Mailadresse & false \\
  \hline
  blankMailTest() & Testet die Überprüfung einer leeren Mailadresse & false \\
  \hline
  \caption{AccountManagerTest}
 \end{longtable}
 
 \begin{longtable}{p{.23\textwidth} | p{.55\textwidth} | p{.12\textwidth}}
\hline
  \textbf{Methode} & \textbf{Test} & \textbf{Ergebnis}\\
  \hline
  videoInfoListTest() & Überprüft die korrekte Ausgabe der VideoInfoList & true \\
  \hline
  downloadTest() & Durchläuft einen Downloadvorgang und überprüft dabei ob ein Inputstream ankommt, mit dem dann eine Testdatei angelegt wird & true \\
  \hline
  deleteTest() & Legt eine Video und Metadata Datei an, die über den Testverlauf gelöscht werden sollen & true \\ 
  \hline
  metadataTest() & Testet die Korrektheit  der Metadateninfomationen, welche über den Klienten angefragt werden & true \\
  \hline
  \caption{VideoManagerTest}
 \end{longtable}
 
  \begin{longtable}{p{.23\textwidth} | p{.55\textwidth} | p{.12\textwidth}}
\hline
  \textbf{Methode} & \textbf{Test} & \textbf{Ergebnis}\\
  \hline
  registerTest() & registriert einen User & true \\
  \hline
  getAccountIdTest() & registriert einen User und vergleicht die id mit der id, die in der Datenbank eingetragen ist & true \\
  \hline
  isVerifiedTest() & registriert und verifiziert einen User & true \\
  \hline
  saveProcessedVideo\newline AndMetaTest() & speichert Video- und Metanamen eines Users nach dessen Registration & true \\
  \hline
  getVideoInfoTest() & testet die VideoInformationen nach Speichern der Video- und Metanamen eines Users und dessen Registration & true \\
  \hline
  getVideoIdBy\newline NameTest() & testet, ob die Methode die VideoId anhand des Namens zurückgibt & true \\
  \hline
  getVideoInfo\newline ListTest() & speichert mehrere Videos zu einem User und prüft, ob die Methode getVideoInfoList ein korrektes Array zurückgibt & true \\
  \hline
  deleteVideo\newline AndMetaTest() & erstellt ein Video zu einem User und löscht es wieder. Test prüft, ob Video in der Datenbank danach noch existiert & true \\
  \hline
  getMeta\newline NameTest() & Testet, ob die Methode getMetaName den Metanamen anhand der VideoId zurückgibt & true \\
  \hline
  setMailTest() & Testet, ob die Mail des aktuellen Accounts nach Aufruf der Methode geändert wurde & true \\
  \hline
  setPasswordTest() & Testet, ob das Passwort des aktuellen Accounts nach Aufruf der Methode geändert wurde & true \\
  \hline
  delete\newline AccountTest() & Testet, ob der aktuelle Account nach Aufruf der Methode noch in der Datenbank existiert & true \\
  \hline
  authenticateTest() & erstellt einen Account und testet, ob die Methode authenticate den aktuellen Account authentifiziert & true \\
  \hline
  isMail\newline ExistingTest() & Testet, ob die Mail schon in der Datenbank existiert & true \\
  \hline
  getSaltTest() & registriert einen Account mit einem bestimmten Salt, um zu testen, ob das Salt in der Datenbank richtig vorhanden ist & true \\
  \hline
  \caption{DatabaseManagerTest}
 \end{longtable}
 
 \begin{longtable}{p{.23\textwidth} | p{.55\textwidth} | p{.12\textwidth}}
\hline
  \textbf{Methode} & \textbf{Test} & \textbf{Ergebnis}\\
  \hline
  nullTest() & Ruft anonymize() mit null als Parameter auf & false\\
  \hline
  validTest() & Ruft anonymize() mit korrekten Parametern auf & true\\
  \hline
  noVideoTest() & Ruft anonymize() auf und versucht ein File zu anonymisieren, das kein Video ist & false\\
  \hline 
  \caption{OpenCVAnonymizeTest}
 \end{longtable}
 
 \begin{longtable}{p{.23\textwidth} | p{.55\textwidth} | p{.12\textwidth}}
\hline
  \textbf{Methode} & \textbf{Test} & \textbf{Ergebnis}\\
  \hline
  nullTest() & Ruft anonymize() mit null als Parameter auf & false\\
  \hline
  validTest() & Ruft anonymize() mit korrekten Parametern auf & true\\
  \hline
  noVideoTest() & Ruft anonymize() auf und versucht ein File zu anonymisieren, das kein korrekten Video ist & false\\
  \hline 
  \caption{OpenCVPythonAnonymizeTest}
 \end{longtable}
 
 \begin{longtable}{p{.23\textwidth} | p{.55\textwidth} | p{.12\textwidth}}
\hline
   \textbf{Methode} & \textbf{Test} & \textbf{Ergebnis}\\
  \hline
  nullTest() & Ruft decrypt() mit null als Parameter auf & false\\
  \hline
  validTest() & Ruft decrypt() mit korrekten Parametern auf & true\\
  \hline 
  \caption{AESDecryptorTest}
 \end{longtable}
 
 \begin{longtable}{p{.23\textwidth} | p{.55\textwidth} | p{.12\textwidth}}
 \hline
 \textbf{Methode} & \textbf{Test} & \textbf{Ergebnis}\\
  \hline
  nullTest() & Ruft decrypt() mit null als Parameter auf & null\\
  \hline
  validTest() & Ruft decrypt() mit korrekten Parametern auf & SecretKey mit Algorighmus 'AES'\\
  \hline 
  invalidFileTest() & Ruft decrypt() auf und versucht ein File zu entschlüsseln, das keinen SecretKey enthält & null\\
  \hline
  \caption{RSADecryptorTest}
 \end{longtable}
 
 \begin{longtable}{p{.23\textwidth} | p{.55\textwidth} | p{.12\textwidth}}
\hline
   \textbf{Methode} & \textbf{Test} & \textbf{Ergebnis}\\
  \hline
  nullTest() & Ruft decrypt() mit null als Parameter auf & false\\
  \hline
  validTest() & Ruft decrypt() mit korrekten Parametern auf & true\\
  \hline 
  \caption{DecryptTest}
 \end{longtable}
 
 \begin{longtable}{p{.23\textwidth} | p{.55\textwidth} | p{.12\textwidth}}
\hline
  \textbf{Methode} & \textbf{Test} & \textbf{Ergebnis}\\
  \hline
  emptyChainTest() & Startet eine VideoProcessingChain mit Chain.EMPTY (Testet Speichern und Löschen von tmp-Daten) & true\\
  \hline
  simpleChainTest() & Startet eine VideoProcessingChain mit Chain.SIMPLE (Testet Chain ohne Anonymisierung) & true\\
  \hline
  normalChainTest() & Startet eine VideoProcessingChain mit Chain.NORMAL (Testet vollständige Chain) & true\\
  \hline
  pythonChainTest() & Startet eine VideoProcessingChain mit Chain.PYTHON (Testet Chain vollständige Chain mit Python Anonymisierung) & true\\
  \hline
  \caption{VideoProcessingChainTest}
 \end{longtable}
 
 \begin{longtable}{p{.23\textwidth} | p{.55\textwidth} | p{.12\textwidth}}
\hline
  \textbf{Methode} & \textbf{Test} & \textbf{Ergebnis}\\
  \hline
  addNullTaskTest() & Ruft addTask() mit Null als Parameter auf & "Not all inputs were given correctly"\\
  \hline
  addValidTaskTest() & Ruft addTask() mit korrekten Parametern auf & "Finished editing Video"\\
  \hline
  shutdownTest() & Ruft shutdown() auf und versucht direkt danach einen Task hinzuzufügen & "Processing module is shut down"\\
  \hline
  \caption{VideoProcessingManagerTest}
 \end{longtable}
\newpage
\subsection{Webinterface}
\subsubsection{Automatisierte Tests}
 \begin{longtable}{p{.23\textwidth} | p{.65\textwidth} | p{.12\textwidth}}
\hline
  \textbf{Methode} & \textbf{Test} & \textbf{Ergebnis}\\
  \hline
  Inhalt & Inhalt & Inhalt\\
  \hline
  \caption{VideoManagerTest}
 \end{longtable}
\subsubsection{Manuelle Tests}

\begin{longtable}{p{.50\textwidth} | p{.50\textwidth}}
\hline
\textbf{Aktion} & \textbf{Reaktion}\\
\hline
Die Seite wird im Browser geöffnet & Es öffnet sich die Login Ansicht\\
\hline
Drücken des Login Buttons ohne/mit ungültiger Eingabe & Ein Popup zeigt einen Fehler an\\
\hline
Drücken des Registrieren Buttons ohne/mit ungültiger Eingabe& Ein Popup zeigt einen Fehler an\\
\hline
Drücken den Registrieren Buttons mit korrekter Eingabe & Ein Popup informiert, den erstellten Account zu verifizieren mittels der gesendeten Mail\\
\hline  
Drücken des Login Buttons ohne klicken des Links in der Mail & Ein Popup zeigt einen Fehler an\\
\hline  
Drücken des Links in der Mail & Eine Seite im Browser teilt erfolgreiche Verifizierung mit\\
\hline
Drücken des Login Buttons nach verifizieren des Accounts & Der Benutzer wird angemeldet und die Video Ansicht wird angezeigt\\
\hline
Drücken des Download Buttons & Es wird ein Dialog geöffnet zum Downloaden der Datei, nach bestätigen wird die Datei heruntergeladen\\
\hline
Drücken des Info Buttons & Es wird ein Popup geöffnet welches die Meta Informationen anzeigt\\
\hline
Drücken des Delete Buttons & Ein Popup zeigt an, dass das Video gelöscht wurde\\
\hline
Drücken des Datenschutz Menüeintrags & Die Datenschutz Informationen werden angezeigt\\
\hline
Drücken des Impressum Menüeintrags & Das PCC Impressum wird angezeigt\\
\hline
Drücken des Account Ansicht Menüeintrags & Die Account Ansicht wird angezeigt\\
\hline
Drücken des ändern Buttons ohne Eingabe & Ein Popup zeigt einen Fehler an\\
\hline
Drücken des ändern Buttons mit falschen Eingaben & Ein Popup zeigt einen Fehler\\
\hline
Drücken des ändern Buttons mit korrekten Eingaben & Ein Popup bestätigt die Änderung und anschließend wird die Login Ansicht angezeigt\\
\hline
Drücken des löschen Buttons & Ein Popup bestätigt, dass der Account gelöscht wurde und die Login Ansicht wird angezeigt\\
\hline
Drücken des Logout Menüeintrags & Die Login Ansicht wird angezeigt\\
\hline

 \end{longtable}
